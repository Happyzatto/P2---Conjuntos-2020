\documentclass[a4paper]{article}
\usepackage{packages}
\title{Introdução à Teoria dos Conjuntos - Prova 2}
\author{Ariel Serranoni Soares da Silva  - Número USP: 7658024\\
Outro Integrante - Número USP:12345678}
\date{20 de Junho de 2020}
\begin{document}
\maketitle
\setcounter{section}{2}
\section{Árvores}

\begin{definition}
  Uma \emph{árvore} é um conjunto ordenado \((T,\leq)\) tal que:
  \begin{enumerate}[(i)]
  \item \(T\) possui um menor elemento;
  \item para cada \(x\in T\), o conjunto \(\{y\in T\,\colon y<x\}\) é bem
    ordenado sob \(\leq\).
  \end{enumerate}
\end{definition}

Os elementos de \(T\) são chamados de \emph{nós}. Se \(x,y\in T\) e \(y<x\),
dizemos que \(y\) é um \emph{antecessor} de \(x\) e que \(x\) é um
\emph{sucessor} de \(y\). O menor elemento de \(T\) é chamado de \emph{raíz}.
Pelo Teorema 3.1 do Capítulo 6, o conjunto bem ordenado \(\{y\in T\,\colon
y<x\}\) de todos os antecessores de \(x\) é isomorfo a um único número ordinal
\(h(x)\), a \emph{altura} de \(x\). O conjunto \(T_\alpha=\{x\in T \,\colon
h(x)=\alpha\}\) é o \(\alpha\)-ésimo \emph{nível} de \(T\). Se \(h(x)\) is é um ordinal
sucessor, então dizemos que \(x\) é um \emph{nó sucessor}, caso contrário
\(x\) é um \emph{nó limite}. O menor \(\alpha\) tal que \(T_\alpha=\varnothing\)
é a \emph{altura} \(h(T)\) da árvore \(T\). 


Um \emph{ramo} de \(T\) é uma cadeia maximal em \(T\). O tipo da ordem
de um ramo \(b\) é o \emph{comprimento} \(\ell(b)\) de \(b\) (sempre menor ou
igual à altura de \(T\)). Um ramo cujo comprimento é igual a \(h(T)\) é chamado
de \emph{cofinal}.

Um subconjunto \(T^\prime\) de \(T\) é uma \emph{subárvore} de \(T\) se para quaisquer
\(x\in T^\prime\) e \(y\in T\) temos que \(y<x\) implica que \(y\in T^\prime\).
Sendo assim, \(T^\prime\) também é uma árvore (quando ordenada por \(\leq\)) e
\(T_\alpha^\prime= T_\alpha \cap T^\prime\) para cada \(\alpha < h(T^\prime)\).
O conjunto \(T^{(\alpha)}=\bigcup_{\beta < \alpha} T_\beta\) é uma subárvore de
\(T\), para cada \(\alpha\leq h(T)\), e \(h(T^{\alpha})=\alpha\). Se \(x\in
T_\alpha\) então \(\{y\in T\,\colon y < x\}\) é um ramo de \(T^{(\alpha)}\) de
comprimento \(\alpha\); entretanto, se \(\alpha\) é um ordinal limite então
\(T^{(\alpha)}\) pode ter outros ramos de comprimento \(\alpha\).

Finalmente, um conjunto \(A\subseteq T\) é uma \emph{anticadeia} em \(T\) se
quaisquer elementos de \(A\) são incomparáveis. Isto é, se \(x,y\in A\) são tais
que \(x\not = y\), então \(x\not < y\) e \(y\not < x\). A seguir, vamos fazer os
Exercícios sugeridos pelo autor:
\begin{exercicio}
  Seja \((T,\leq)\) uma árvore. Mostre que:
  \begin{enumerate}[(i)]
  \item O único elemento \(r\in T\) tal que \(h(r)=0\) é a raíz. Em particular,
    \(T_0\not=\varnothing\);
  \item Se \(\alpha\not = \beta\), então \(T_\alpha\cap T_\beta =\varnothing\);
  \item \(T=T^{(h(T))}=\bigcup_{\alpha\leq h(T)}T\alpha\);
  \item \(x\in T\) é um nó sucessor se, e somente se existe \(y\in T\) tal que
    \(y < x\) e não existe \(z\) tal que \(y<z<x\). Se \(x\) é um nó sucessor
    então tal \(y\) é único e é chamado de \emph{antecessor imediato} de \(x\),
    e \(x\) é chamado de \emph{sucessor imediato} de \(y\). (Cada nó possui um
    único antecessor imediato, mas pode possuir múltiplos sucessores imediatos)
    \item Se \(y< x\) então existe um único \(z\in T\) tal que \(x< x\leq y\) e
      \(z\) é um sucessor imediato de \(x\);
    \item \(h(T)=\sup\{\alpha +1\,\colon
      T_\alpha\not=\varnothing\}=\sup\{h(x)+1\,\colon x\in T\}.\)
  \end{enumerate}

\end{exercicio}

  \begin{exercicio}
    Mostre que:
    \begin{enumerate}[(i)]
    \item Cada cadeia em \(T\) é bem-ordenada;
    \item Se \(b\) é um ramo de \(T\) e \(x\in b\), e \(y<x\), então \(y\in b\);
    \item Se \(b\) é um ramo em \(T\), então \(\card{b\cap T_\alpha}=1\) para
      \(\alpha <\ell (b)\) e \(\card{b\cap T_\alpha}=0\) para \(\alpha
      >\ell (b)\). Conclua que \(\ell (b)\leq h(T)\);
    \item \(h(T)=\sup\{\ell (b)\,\colon b \text{ é um ramo de } T\}\);
    \item \(T_\alpha\) é uma anticadeia para cada \(\alpha < h(T)\).
     \end{enumerate}
  \end{exercicio}


 
\end{document}